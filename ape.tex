\subsection{CAM6 aqua-planet simulations}

The 'CESM simpler models' effort also supports aqua-planet configurations (Medeiros et al. 2016). The aqua-planet configurations (Neale and Hoskins 2001) refer to an ocean covered planet with no axial tilt -- a planet in a perpetual equinox and devoid of continents. The aqua-planet compsets designate a fixed, zonally-symmetric SST distribution, modeled after the present day SST distribution on Earth ('QOBS' in Neale and Hoskins 2001). The lack of a seasonal cycle allows one to compute robust statistics from a shorter simulation, and the absence of land removes any influence of discretized topography or interactions with the land model from the simulations. The aqua-planet configurations are therefore an indispensable tool for global atmospheric model developers.

To understand how the design choices adopted by CAM-SE influence the aqua-planet solutions compared to its predecessor, CAM-HOMME, we ran a pair of simulations using the CAM6 physics package for 4.5 years. Figure X shows the total kinetic energy spectrum at the 200 hPa level in the two simulations. Compared to CAM-HOMME, the slope of the kinetic energy spectrum in CAM-SE is shallower for wavenumbers larger than 30, bringing the solutions closer to the empirically and theoretically determined slope of -3. (IS -3 TRUE FOR AQUA-PLANETS?). The increased kinetic energy at smaller scales is primarily a result of the smaller divergence damping coefficient used in CAM-SE (not shown).

Figure X shows the zonally averaged total precipitation rate in CAM-HOMME (purple) and CAM-SE (red) averaged over the final 4 years of the simulations. The differences between the two simulations is provided as the purple curve on the bottom plot of Figure X. CAM-SE has increased precipitation at the equator, with a peak difference of 3 mm/day. An approximately 1 mm/day reduction in precipitation occurs on the flanks of the equator (Figure X). CAM-HOMME has twin Inter-Tropical Convergence Zones (ITCZ) straddling the equator, which appears absent from the CAM-SE simulations. The total precipitation rate in the model is the sum of the precipitation due to the deep convective parameterization, and that due to CLUBB, primarily from the large-scale condensation routine within CLUBB. (DOES PRECC HAVE CLUBB SHALLOW CONVECTION IN IT?) The deep convective precipitation rate indicates that the double-ITCZ does exist in CAM-SE (not shown), but is hidden from the total precipitation rate due to the increase in large-scale condensation at the equator.

An analysis of the vertically integrated, zonally averaged dry static energy budget indicates that the changes in the precipitation rate are balanced by the anomalous dry static energy flux convergence due to the increase in the mean resolved vertical upward motion at the equator ($\Delta{H_{dyn}}$ from Muller and O'Gorman 2011; not shown). This balance also holds for the reduction in precipitation rate on the flanks of the equator (not shown). It is likely that an increase in resolved vertical upward motion in CAM-SE drives the increase in large-scale condensation rate observed in CAM-SE, consistent with a prior analysis of CAM-HOMME (O'Brien et al. 2016). In this scenario, greater downward motion observed on the flanks of the equator are simply compensating for the increased mass flux at the equator, which then acts to stabilize the column and reduce the precipitation rate locally.

The authors have identified three design aspects of CAM-SE that explain the large changes in precipitation observed in the Tropics. These three design aspects are: the use of a lower divergence damping coefficient, a thermodynamically consistent calculation of $c_p$ (eqn. 21) and the removal of the limiter in the PPM reconstruction in the vertical remapping routine. Figure X shows the total precipitation rate from three additional CAM-SE simulations, each simulation having one of the three modifications reverted back to the CAM-HOMME design. By using the larger divergence damping coefficients from CAM-HOMME in CAM-SE ('CAM-SE-oldvisc'), the precipitation rate at the equator is reduced by about 1 mm/day compared with CAM-SE. The CAM-SE simulation in which the PPM limiter is turned on (CAM-SE-ppmlimiter), as it is in CAM-HOMME, results in a dramatic increase in equatorial precipitation of about 3 mm/day compared with CAM-SE (Figure X). And finally, through reverting the definition of $c_p$ back to the CAM-HOMME definition ($c_p = c_p^{(d)}$; 'CAM-SE-cpcnst'), the near equatorial precipitation rates are dramatically reduced by about 3 mm/day (Figure X). A forth simulation was performed containing all three of the aforementioned modifications ('CAM-SE-all'). The simulated total precipitation rates in CAM-SE-all are indistinguishable from the CAM-HOMME simulation (Figure X).  

The influence of the new definition of $c_p$ and the PPM limiter on Tropical precipitation are not nearly as intuitive as the influence of the divergence damping coefficient. An increase in divergence damping reduces the magnitude of vertical pressure velocities, likely explaining the reduction in precipitation in that simulation. In our aqua-planet simulations, the ITCZ in part, consists of under-resolved, deep convective towers simulated by the resolved dynamics, consistent with a prior study of CAM-HOMME (Herrington and Reed 2017). The base of the convective towers often originate within the boundary layer, with convergent flow driving resolved upward mass fluxes through the cloud base. In the CAM-SE-ppmlimiter simulation, the magnitude of the velocities in the lowest model level of the equatorial convergence zone are larger. The authors speculate that the limiter will select a larger magnitude horizontal wind in the lowest model levels in a convergent flow regime, resulting in greater convergence, vertical motion and therefore precipitation. 

To explain how the limiter works, consider a vertical profile of the horizontal wind in the aqua-planet simulations across the lowest model level, u(k1), and the second lowest model level, u(k2). In the equatorial regions, there is a monotonic increase in the magnitude of the wind with height, across the two levels. If level k1 is undergoing horizontal mass convergence, the model level will be raised. To find the mean wind on the new grid, u(l1), one integrates over the PPM subgrid reconstruction to within a region that is occupied by k2. Since the magnitude of the wind increases with height, the contribution of the mean wind in u(l1) from the region occupied by k2 will always have a smaller magnitude than u(k2), which is computed by integrating to the top of k2, where the magnitude of the winds are larger. The limiter forces the contribution of the mean wind to u(l1) from k2 to be equal to u(k2). In this scenario, the limiter will result in larger magnitude winds in a convergent flow regime in the lowest model level, facilitating greater mass convergence during the next time-step.

\begin{figure}[h]
\centering
\includegraphics[width=20pc]{figs/dzonal_prect.pdf}
\caption{(Upper)(lower)}
\label{fig:dzonal}
\end{figure}

