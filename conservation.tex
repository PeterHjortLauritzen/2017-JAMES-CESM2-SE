\subsubsection{Conservation}
To assess the AAM conservation properties of the CAM-SE dynamical core, the diagnostics used in \cite{LBDL2014JAMES} are applied to the Aqua-planet simulation. In other words, at various places in the dynamical core the column integrated wind and mass AAM are written to history tapes and the total integrals of AAM are computed as a post-processing step. The torques are obtained by subtracting the global integrals of AAM divided by the time-increment between the outputs of the AAM. For details on the discretized AAM diagnostics see \cite{LBDL2014JAMES}. These diagnostic outputs are part of the CESM2 release and are controlled with namelist variables.

As discussed in Section \ref{sec:aam} the total AAM torque from the dynamical core, in the absence of topography, should be small compared to the torque from the parameterizations. Figure \ref{fig:aam} shows that that is indeed the case for CAM-SE. The spurious torques from the dynamical core are approximately a factor 100 smaller than the physical torques from the parameterizations.

Similarly to the AAM diagnostics, the total column-integrated moist energy is output at various places in the dynamical core and physics parameterizations. The dynamical core conserves the moist total energy, equation \eqref{eq:comprehensice_energy}, to about 0.1$W/m^2$. The frictional heating term described in section \ref{sec:frictional_heating} is approximately 0.4$W/m^2$ and hence an important term for total moist energy conservation. As mentioned in section \ref{sec:aam} the CAM physics energy fixer enforces a different energy than the comprehensive moist energy. This inconsistency should be removed in future CAM versions but it is, however, not a trivial modification to the CAM physics package. The discrepancy between the two definitions of energy is approximately 0.5$W/m^2$ \citep[similar to what ][ found when just including the correct heat capacity for water vapor in the total energy equation]{T2011LNCSEb}. A detailed energy analysis will be the subject of a future publication.
